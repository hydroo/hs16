\section{Additional Literature}

The inclined reader may refer to one or more of the works listed below to follow are more in-depth introduction to modern C++:

\begin{itemize}
\item Stanley B. Lippmann, Josee Lajoie, Barbara E. Moo: \textit{C++ Primer}, Addison-Wesley, 2012. The intended audience of this book are beginners, both in programming and C++. It follows a high-level approach; for example it starts with containers before explaining dynamic memory allocation.
\item Bjarne Stroustrup: \textit{A Tour of C++}, Addison-Wesley, 2014. This book targets people who have programmed before (not necessarily in C++) and provides them with a broad introduction to the features found in modern C++.
\item Bjarne Stroustrup: \textit{The C++ Programming Language}, Addison-Wesley, 2013. With this work the creator of the C++ programming language published an exhaustive introduction to modern C++, covering the language and its standard library in great detail.
\item Scott Meyers: \textit{Effective Modern C++}, O'Reilly, 2014. Targeting an experienced audience, Scott Meyers provides a lot of optimization hints with this work.
\item Anthony Williams: \textit{C++ Concurrency in Action}, Manning, 2012. As the title says this book focuses on modern C++ concurrency mechanisms.
\end{itemize}

Additionally, the following websites are valuable sources of information:

\begin{itemize}
\item Bjarne Stroustrup: \textit{C++11 FAQ}, \url{http://www.stroustrup.com/C++11FAQ.html}. This page explains the frequently questioned design decisions of the C++11 standard.
\item Herb Sutter: \textit{Guru of the Week}, \url{https://www.herbsutter.com/gotw/}. In this weekly weblog series Herb Sutter explains C++ features in great depth, usually by providing a puzzle and explaining the solution a week later.
\end{itemize}