\section{Outlook On C++17}

\subsection{The Standardization Process}

\subsection{\texttt{if}-Statement With Initializer}
\subsection{Structured Bindings}
\subsection{Parallel Algorithms}

\subsubsection{Parallel \texttt{find}}

With the introduction of C++17 \texttt{std::find} will have the following signatures:

\begin{lstlisting}[language=C++]
template <class InputIt, class T>
InputIt find(InputIt first, InputIt last, const T& value);

template <class ExecutionPolicy, class InputIt, class T>
InputIt find(ExecutionPolicy&& policy, InputIt first, InputIt last,
    const T& value);
\end{lstlisting}

While the first variant will not be parallelized, the second may be executed with the parallel execution policy type. This raises the question how a parallel \texttt{std::find} is implemented internally. As of the writing of this document most of the commonly available standard libraries do not support the proposed parallel algorithms; however, Microsoft provided a reference implementation. Internally, their parallel \texttt{find} works in a similar fashion to the code in Listing~\ref{lst:find_impl}: After executing the function the runtime will spawn a number of threads and split the container amongst them. Each thread then processes his part of the container and sets a shared atomic variable if the value in question was found, unless a better match was found by another thread already.

\begin{lstlisting}[caption={A Microsoft-inspired parallel find implementation},label={lst:find_impl},language=C++]
template <class InputIt, class T>
InputIt find_impl(std::execution::par, InputIt first, InputIt last,
    const T& value)
{
    // container size
    auto size = std::distance(first, last);

    // shared among threads
    using difference_type = typename std::iterator_traits<InputIt>::difference_type;
    std::atomic<difference_type> state(size);
    
    /*
     * Omitted: spawn a number of threads, starting at different
     * positions between [first, last). The work sizes are
     * stored in "count" and the thread-local starting position
     * in "begin"
     */
    auto dist = std::distance(first, begin);
    for(auto pos = 0; pos < count; ++pos, ++begin)
    {
        if(value == *begin)
        {
            auto cur = dist + pos;
            auto old = state.load(std::memory_order_relaxed);
            
            do
            {
                // do not replace if there is a closer match already found
                if(old < cur)
                    break;
            } while(!state.compare_exchange_strong(old, cur, std::memory_order_relaxed));
            
            break;
        }
    }
    
    /*
     * Omitted: join threads
     */
    auto pos = state.load(std::memory_order_relaxed);
    if(pos != size)
    {
        std::advance(first, pos);
        return first;
    }
    
    return last;
}
\end{lstlisting}